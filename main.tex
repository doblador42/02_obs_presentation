\documentclass[aspectratio=169]{beamer}
\usepackage{fontspec}
\usepackage{hyperref}
\usepackage{ccicons}
\protrudechars=2 % or \pdfprotrudechars=2 and
\adjustspacing=2 %    \pdfadjustspacing=2 with luatex < v0.85
\usepackage{amsmath}
\usepackage{graphicx}
\usepackage[table]{xcolor} % Για χρωματισμούς στις γραμμές και τις στήλες
\usepackage{xcolor}
\usepackage{polyglossia}
\usepackage{csquotes}
\usepackage[symbol]{footmisc}
\setmainlanguage{greek}
\setotherlanguage{english}
\setmainfont[Numbers={OldStyle, Proportional},
Ligatures=TeX]{Linux Libertine O}
\newfontfamily\greekfont{Linux Libertine O}
\newfontfamily\englishfont{Linux Libertine O}
\newfontfamily\greekfontsf{Linux Libertine O}
\usetheme{Madrid}
\usecolortheme{beaver}
\title{Εισαγωγή στο λογισμικό της διαδικτυακής αναμετάδοσης}
\subtitle{\href{https://obsproject.com/}{OBS (Open Broadcaster Software)} και πλατφόρμες διαδικτυακής αναμετάδοσης} 
\author{Ερμής Δούλος (\textit{dit17046@uop.gr})}
\begin{document}
\section{intro}
\begin{frame}
  \titlepage
  \begin{center}
    % GitHub Link
    \href{https://github.com/doblador42}{\textbf{Github Profile:}}\\
    % QR Code
    \includegraphics[width=0.13\textwidth]{images/qrcode.png}
  \end{center}
\end{frame}
\begin{frame}{Περιεχόμενα}
  \tableofcontents
\end{frame}

\section{Επισκόπηση}
\begin{frame}{Επισκόπηση}
  \begin{block}{Περιεχόμενα}
    \begin{itemize}
      \item Ανασκόπηση του λογισμικού μιας τηλεμετάδοσης
      \item Εισαγωγή στο \href{https://obsproject.com/}{OBS}
            \begin{itemize}
              \item Εγκατάσταση
              \item Ρύθμιση
              \item Χρήση
                    \begin{itemize}
                      \item Πηγές
                      \item Σκηνές
                      \item Επιλογές διάταξης
                    \end{itemize}
            \end{itemize}
      \item Πλατφόρμες αναμετάδοσης
            \begin{itemize}
              \item \href{https://www.youtube.com/}{YouTube}
              \item \href{https://zoom.us/}{Zoom}
              \item \href{https://www.webex.com/}{Webex}
              \item \href{https://www.microsoft.com/el-gr/microsoft-teams/group-chat-software}{Teams}
            \end{itemize}
      \item Παραδείγματα
    \end{itemize}
  \end{block}
\end{frame}
\subsection{διάγραμμα διάταξης αναμετάδοσης}
\begin{frame}{Διάγραμμα όλης της αναμετάδοσης}
  \begin{figure}
    \includegraphics[width=0.8\textwidth]{images/diagram.png}
    \caption{Διάγραμμα όλης της αναμετάδοσης}
    \label{fig:diagram}
  \end{figure}
\end{frame}
\subsection{διάγραμμα του αντικειμένου μας για σήμερα}
\begin{frame}{Με τι θα ασχοληθούμε σε αυτό το session;}
  \begin{block}{Η διάταξη που θα μας απασχολήσει σήμερα:}
    \begin{figure}
      % full fledged figure with caption and label
      \includegraphics[width=0.42\textwidth]{images/laptop.png}
      \caption{Διάταξη που μας απασχολεί στο σημερινό session}
      \label{fig:laptop}
    \end{figure}
  \end{block}
\end{frame}
\section{Πρόγραμμα για live-streaming;}
\subsection{τι είναι αυτά που ζητάμε;}
\begin{frame}{Τι χρειαζόμαστε για μια διαδικτυακή αναμετάδοση;}
  \begin{block}{Πρόγραμμα αναμετάδοσης}
    \begin{itemize}
      \item Εύκολη χρήση\footnote{Χωρίς πολύπλοκες ρυθμίσεις και προσβάσιμο και σε μη ειδικούς}
      \item Ευελιξία\footnote{Δυνατότητα προσθήκης πολλαπλών πηγών και παραμετροποίησης της εμφάνισης τους}
      \item Επαγγελματική ποιότητα\footnote{Καλή ποιότητα εικόνας και ήχου χωρίς διαλείψεις και εκπλήξεις}
      \item Ελεύθερο λογισμικό\footnote{Διότι είναι δωρεάν και ανοιχτού κώδικα}
    \end{itemize}
  \end{block}
\end{frame}
\subsection{Δημοφιλή προγράμματα αναμετάδοσης}
\begin{frame}{Λογισμικό αναμετάδοσης}
  \begin{columns}
    \begin{column}{0.4\textwidth}
      \begin{exampleblock}{Προτεινόμενο λογισμικό}
        \begin{itemize}
          \item \textbf{\href{https://obsproject.com/}{OBS (Open Broadcaster Software)}}
          \item \href{https://www.xsplit.com/}{XSplit}
          \item \href{https://streamlabs.com/}{Streamlabs OBS}
          \item \href{https://www.telestream.net/wirecast/}{Wirecast}
          \item \href{https://www.vmix.com/}{vMix}
        \end{itemize}
      \end{exampleblock}
    \end{column}
    \begin{column}{0.6\textwidth}
      \begin{center}
        \includegraphics[width=0.9\textwidth]{images/meme.jpg}
      \end{center}
    \end{column}
  \end{columns}
\end{frame}
\section{Το OBS}
\subsection{Εισαγωγικές πληροφορίες;}
\begin{frame}{Τι είναι το OBS και γιατί το χρειαζόμαστε;}
  \begin{block}{Ορισμός}
    Το \textbf{\href{https://obsproject.com/}{Open Broadcaster Software}} είναι ένα λογισμικό ανοιχτού κώδικα που χρησιμοποιείται
    για τοπική καταγραφή και αναμετάδοση περιεχομένου σε πλατφόρμες όπως το \href{https://www.youtube.com/}{YouTube},
    το \href{https://www.twitch.tv/}{Twitch}, το \href{https://www.facebook.com/}{Facebook} και το \href{https://twitter.com/}{Twitter}.
  \end{block}
  \begin{block}{Χρησιμότητά του για εμάς}
    Για εμάς το \textbf{OBS} είναι ένα εργαλείο που μας επιτρέπει να μοιραστούμε \textbf{διαδικτυακά} το τι γίνεται μέσα στο \textbf{αμφιθέατρο}
    της σχολής μας, χρησιμοποιώντας τις \textbf{κάμερες}, τα \textbf{μικρόφωνα} όλης της αίθουσας και τις \textbf{παρουσιάσεις} που γίνονται σε αυτό.
    Κύρια πλεονεκτήματά του είναι πως είναι \textbf{δωρεάν} και \textbf{ανοιχτού κώδικα}, καθώς επίσης πως είναι \textbf{δοκιμασμένο} και \textbf{αξιόπιστο}.
  \end{block}
\end{frame}
\subsection{Εγκατάσταση}
% Installation Slide
\begin{frame}{Εγκατάσταση του OBS}
  \begin{columns}
    \begin{column}{0.8\textwidth}
      \begin{block}{Ελάχιστες Απαιτήσεις Συστήματος}
        \begin{itemize}
          \item \textbf{Λειτουργικό Σύστημα:} Windows 10 ή νεότερο, macOS 10.13 ή νεότερο, Linux
          \item \textbf{Επεξεργαστής:} Intel Core i5 6ης γενιάς ή νεότερος ή αντίστοιχος AMD
          \item \textbf{Μνήμη RAM:} 8GB ή περισσότερο
          \item \textbf{Κάρτα Γραφικών:} DirectX 11.1 / OpenGL 3.3 capable ή νεότερο\footnote{Μερικές φορές όπως στη δική μας περίπτωση, η ενσωματωμένη κάρτα γραφικών είναι αρκετή.}
        \end{itemize}
      \end{block}
      \begin{block}{Εγκατάσταση}
        Κατεβάστε το \textbf{OBS} από την επίσημη ιστοσελίδα: \href{https://obsproject.com/}{obsproject.com}
      \end{block}
    \end{column}
    \begin{column}{0.2\textwidth}
      \begin{center}
        \includegraphics[width=0.95\textwidth]{images/obs2.png}
      \end{center}
    \end{column}
  \end{columns}
\end{frame}
\subsection{Διεπαφή και ρύθμιση}
\begin{frame}[allowframebreaks]{Η διεπαφή του OBS}
  \begin{block}{Αρχική οθόνη}
    \begin{figure}
      \includegraphics[width=0.38\textwidth]{images/screenshot_obs.png}
      \caption{Η κύρια οθόνη του OBS Studio}
      \label{fig:obs_interface}
    \end{figure}
  \end{block}
  \begin{block}{Επιλογές στη βασική οθόνη}
    \begin{figure}
      \includegraphics[width=0.8\textwidth]{images/obs_tools.png}
      \caption{Η οθόνη ρύθμισης του OBS Studio}
      \label{fig:obs_interface2}
    \end{figure}
  \end{block}
\end{frame}
\section{Συνάντηση δια-ζώσης}
\begin{frame}[allowframebreaks]{Εξάσκηση στο OBS}
  \begin{alertblock}{Πότε και πού}
    \begin{itemize}
      \item \textbf{Ημερομηνία:} Τρίτη, 3 Δεκεμβρίου 2024
      \item \textbf{Ώρα:} 3:00 μ.μ.
      \item \textbf{Τοποθεσία:}
            \begin{itemize}
              \item Εργαστήριο CNA LAB
              \item Πάνω όροφος
              \item Κάτω κτήριο Σχολής Επιστημών και Τεχνολογίας
            \end{itemize}
    \end{itemize}
  \end{alertblock}
  \begin{exampleblock}{Τι θα κάνουμε την Τρίτη;}
    Θα δούμε αναλυτικά:
    \begin{itemize}
      \item Πώς να προσθέσουμε και να διαχειριστούμε πηγές (κάμερες, μικρόφωνα, εικόνες, κ.λπ.)
      \item Πώς να δημιουργήσουμε και να εναλάσσουμε σκηνές
      \item Πώς να ρυθμίσουμε τις παραμέτρους για βέλτιστη ποιότητα αναμετάδοσης
      \item Πώς να χρησιμοποιήσουμε τα εργαλεία του OBS για να προσθέσουμε γραφικά και κείμενο
      \item Πώς να ενσωματώσουμε το OBS με πλατφόρμες αναμετάδοσης όπως το YouTube και το Zoom
    \end{itemize}
  \end{exampleblock}
  \begin{alertblock}
    {Προσοχή!}
    \begin{itemize}
      \item Προτείνεται να φέρετε το δικό σας φορητό υπολογιστή
      \item Θα χρειαστείτε λογαριασμό στο YouTube για την αναμετάδοση
      \item Θα χρειαστείτε λογαριασμό στο Zoom για την αναμετάδοση
    \end{itemize}
  \end{alertblock}
\end{frame}

\begin{frame}{Ευχαριστώ για την προσοχή σας!}
  \begin{columns}
    \begin{column}{0.4\textwidth}
      \begin{center}
        \Huge{Ερωτήσεις;}
      \end{center}
    \end{column}
    \begin{column}{0.6\textwidth}
      \begin{center}
        \includegraphics[width=0.9\textheight]{images/xkcd.png}
      \end{center}
    \end{column}
  \end{columns}
\end{frame}
% License Slide
\begin{frame}{Άδεια Χρήσης}
  \begin{center}
    \ccbysa
  \end{center}
  \begin{center}
    Άδεια \href{https://creativecommons.org/licenses/by/4.0/}{Creative Commons Αναφορά Δημιουργού 4.0 Διεθνές (CC BY 4.0)}.  Το παρόν διατίθεται υπό την
    \ccbysa
  \end{center}
  \begin{exampleblock}{Επιτρέπεται στον αποδέκτη:}
    \begin{itemize}
      \item Να μοιραστεί το έργο με άλλους.
      \item Να τροποποιήσει το έργο για προσωπική ή εμπορική χρήση.
      \item Να χρησιμοποιήσει το έργο σε παρουσιάσεις ή δημοσιεύσεις.
      \item Να αναφέρει τον δημιουργό του έργου όταν το χρησιμοποιεί.
    \end{itemize}
  \end{exampleblock}
\end{frame}
\end{document}